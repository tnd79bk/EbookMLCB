%!TEX root = ../../book_ML.tex

\chapter{Các khái niệm cơ bản}



Một thuật toán machine learning là một thuật toán có khả năng {học tập} từ dữ
liệu. Vậy thực sự ``học tập'' ở đây có nghĩa như thế nào? Theo
Mitchell~\cite{mitchell1997machine}, ``\textit{A computer program is said to learn from
\textit{experience} $E$ with respect to some \textit{tasks} $T$ and
\textit{performance measure} $P$, if its performance at tasks in $T$, as
measured by $P$, improves with experience $E$.}
''

Tạm dịch: 

\begin{mydef}{Học (chương trình máy tính)}{learning}
Một chương trình máy tính được gọi là ``học'' từ \textit{kinh nghiệm} $E$
để hoàn thành \textit{nhiệm vụ} $T$ với hiệu quả được đo bằng \textit{phép đánh
giá} $P$ nếu hiệu quả của nó khi thực hiện nhiệm vụ $T$ khi được đánh giá bởi
$P$ cải thiện theo kinh nghiệm $E$.
\end{mydef}

Lấy ví dụ về một chương trình máy tính có khả năng tự chơi cờ vây. Chương trình này tự học từ các ván cờ đã chơi trước đó của con người để tính toán ra các chiến thuật hợp lý nhất. Mục đích của việc học này là tạo ra một chương trình có khả năng giành phần thắng cao. Chương trình này cũng có thể tự cải thiện khả năng của mình bằng cách chơi hàng triệu ván cờ với chính nó. Trong ví dụ này, chương trình máy tính có nhiệm vụ chơi cờ vây thông qua kinh nghiệm là {các ván cờ đã chơi} với chính nó và của con người. Phép đánh giá ở đây chính là khả năng giành chiến thắng của chương trình.

Để xây dựng một chương trình máy tính có khả năng học, ta cần xác định rõ ba yếu tố: nhiệm vụ, phép đánh giá, và nguồn dữ liệu huấn luyện. 

Trong chương này, chúng ta sẽ tìm hiểu các yếu tố này thông qua các ví dụ.

\index{Nhiệm vụ task}
\section{Nhiệm vụ $T$}

% Trong bài toán phân loại một bức ảnh, việc phân loại được gọi là \textit{task}.
% Trong bài toán phân nhóm dữ liệu, việc phân nhóm được gọi là \textit{task}.
% Nếu ta muốn một con robot có thể tự học cách đi lại, thì việc đi lại được gọi là
% \textit{task}. 

\index{điểm dữ liệu -- data point}
\index{vector đặc trưng -- feature vector}
\index{đặc trưng -- feature}
\index{tensor}
Các \textit{nhiệm vụ} trong machine learning được mô tả thông qua việc
một hệ thống xử lý một điểm dữ liệu đầu vào như thế nào. 

Một điểm dữ liệu có thể là một bức ảnh, một đoạn âm thanh, một văn bản, hoặc một
tập các hành vi của người dùng trên Internet. Để chương trình máy tính có thể
học được, các điểm dữ liệu thường được đưa về dạng tập hợp các con số mà mỗi số
được gọi là một \textit{đặc trưng}.


Có những loại dữ liệu được biểu diễn dưới dạng ma trận hoặc mảng nhiều chiều.
Một bức ảnh xám có thể được coi là một ma trận mà mỗi phần tử là giá trị độ sáng
của điểm ảnh tương ứng. Một bức ảnh màu ba kênh đỏ, lục, và lam có thể được biểu
diễn bởi một mảng ba chiều. Trong cuốn sách này, các điểm dữ liệu đều được biểu
diễn dưới dạng mảng một chiều, còn được gọi là \textit{vector đặc trưng}. Vector đặc trưng của một điểm dữ liệu thường được ký hiệu là $\bx \in \R^d$ trong đó $d$ là số lượng đặc trưng. Các mảng nhiều chiều được hiểu là đã bị \textit{vector hoá} thành mảng một chiều. Kỹ thuật xây dựng vector đặc trưng cho dữ liệu được trình bày cụ thể hơn trong Chương~\ref{cha:feature}.

% Tập hợp này thường được biểu diễn dưới dạng
% một \textit{vetor đặc trưng} \footnote{Có những loại dữ liệu được biểu diễn dưới
% dạng ma trận hoặc mảng nhiều chiều. Một bức ảnh xám có thể được coi là một ma
% trận mà mỗi phần tử là giá trị độ sáng của điểm ảnh tương ứng. Một bức ảnh màu
% ba kênh đỏ, lục, và lam có thể được biểu diễn bởi một mảng ba chiều. Trong cuốn
% sách này, các điểm dữ liệu đều được biểu diễn dưới dạng vector, các mảng nhiều
% chiều }

% Trong bài toán phân loại ảnh, mỗi ảnh là một điểm dữ liệu.
% Trong bài toán phân nhóm khách hàng, mỗi khách hàng là một điểm dữ liệu. Trong
% bài toán xác định một tin nhắn có là rác hay không, mỗi tin nhắn là một điểm dữ
% liệu. Mỗi điểm dữ liệu bao gồm nhiều \textit{đặc trưng} (\textit{feature}) khác
% nhau, mỗi feature thường được biểu diễn dưới dạng một con số. Chúng ta thường
% biểu diễn một điểm dữ liệu như một vector\footnote{Có những loại dữ liệu không
% được biểu diễn dưới dạng một vector mà có thể là một ma trận -- khi giữ nguyên một
% bức ảnh trong không gian hai chiều, hoặc một \textit{tensor} -- mảng nhiều
% chiều -- khi xem các bức ảnh với nhiều channel khác nhau. Trong cuốn sách này,
% chúng ta chỉ xét các điểm dữ liệu dưới dạng vector, hoặc \textit{vector hoá}
% (\textit{vectorization}) các điểm dữ liệu nhiều chiều.} $\bx \in \R^d$ trong đó
% mỗi phần tử $x_i$ là một đặc trưng, vector này thường được gọi là \textit{vector
% đặc trưng} (\textit{feature vector}). Ví dụ, trong một bức ảnh, mỗi giá trị của
% một điểm ảnh có thể coi là một đặc trưng, vector chứa toàn bộ giá trị các pixel
% của ảnh có thể coi là một vector đặc trưng. Chương~\ref{cha:feature} sẽ bàn sâu
% thêm về vector đặc trưng của dữ liệu.

Nhiều nhiệm vụ phức tạp có thể được giải quyết bằng machine learning. Dưới đây
là một số nhiệm vụ phổ biến.

% \begin{itemize}

\subsection{Phân loại -- classification}
\index{Phân loại -- classification}
\textit{Phân loại}~\footnote{có tài liệu gọi là \textit{phân lớp}} là một
trong những bài toán được nghiên cứu nhiều nhất trong machine learning. Trong
bài toán này, chương trình được yêu cầu xác định loại/nhãn\footnote{\textit{class} hoặc \textit{label} trong tiếng Anh.} của một điểm dữ liệu trong số $C$ nhãn khác nhau.

% Nhãn này thường là một phần
% tử trong một tập hợp có $C$ phần tử khác nhau. Mỗi phần tử trong tập hợp này
% được gọi là một \textit{lớp} (\textit{class}), và thường được đánh số từ $1$ đến
% $C$. Để giải bài toán này, ta thường phải xây dựng một hàm số $f: \R^d
% \rightarrow \{1, 2, \dots, C\}$. Khi $y = f(\bx)$, mô hình gán cho một điểm dữ
% liệu được mô tả bởi vector đặc trưng $\bx$ một nhãn được xác định bởi số $y$.


\textbf{Ví dụ:} Trong nhận dạng chữ số viết tay, ta có ảnh của hàng nghìn ví
dụ của mỗi chữ số được viết bởi nhiều người khác nhau. Các bức ảnh này cùng với
nhãn của chúng được đưa vào một thuật toán machine learning. Sau khi thuật toán
này \textit{học} được một mô hình, tức một hàm số mà đầu vào là một bức ảnh và
đầu ra là một chữ số, khi nhận được một bức ảnh mới mà mô hình \textbf{chưa nhìn
thấy bao giờ}, nó sẽ dự đoán bức ảnh đó chứa chữ số nào.
Ví dụ này khá giống với cách học của con người khi còn nhỏ. Ta đưa bảng chữ cái
cho một đứa trẻ và chỉ cho chúng đây là chữ A, đây là chữ B. Sau một vài lần
được dạy thì trẻ có thể nhận biết được đâu là chữ A, đâu là chữ B mà chúng chưa nhìn thấy bao giờ.

Có một biến thể nhỏ ở đầu ra của hàm số $f(\bx)$ khi đầu ra không phải là một số
mà là một vector $\by \in \R^C$ trong đó $y_c$ chỉ ra xác suất để điểm dữ liệu
$\bx$ rơi vào lớp thứ $c$. Lớp được chọn cuối cùng là lớp có xác suất rơi vào là
cao nhất. Việc sử dụng xác suất này đôi khi rất quan trọng, nó giúp chỉ ra
\textit{độ chắc chắn} (\textit{confidence}) của mô hình. Nếu xác suất cao nhất
là cao hơn nhiều so với các xác suất còn lại, ta nói mô hình có độ chắn chắn là
cao khi phân lớp điểm dữ liệu $\bx$. Ngược lại, nếu độ chênh lệch giữa xác suất
cao nhất và các xác suất tiếp theo là nhỏ, thì khả năng mô hình đã phân loại
nhầm là cao hơn.

\subsection{Regression}
\index{regression -- hồi quy}
Nếu nhãn không được chia thành các nhóm mà là các giá trị thực (có thể vô hạn)
thì bài toán được gọi là \textit{hồi quy}, một số tài liệu gọi là \textit{tiên
lượng} (\textit{regression}). Trong bài toán này, ta
cần xây dựng một hàm số $f: \R^d \rightarrow \R$. 

\textbf{Ví dụ 1:} Ước lượng một căn nhà rộng $x ~ \text{m}^2$, có $y$ phòng ngủ
và cách trung tâm thành phố $z~ \text{km}$ sẽ có giá khoảng bao nhiêu?

\textbf{Ví dụ 2:} Microsoft có một ứng dụng dự đoán giới tính và tuổi dựa trên
khuôn mặt (\url{http://how-old.net/}). Phần dự đoán giới tính có thể được coi là
một thuật toán classification, phần dự đoán tuổi có thể coi là một thuật toán
{regression}. {Chú ý rằng phần dự đoán tuổi cũng có thể coi là {classification}
nếu ta coi tuổi là một số nguyên dương không lớn hơn 150, chúng ta sẽ có 150
class (lớp) khác nhau.}

Bài toán regression có thể mở rộng ra việc dự đoán nhiều đầu ra cùng một lúc,
khi đó, hàm cần tìm sẽ là $f: \R^d \rightarrow \R^m$. Một ví dụ là bài toán
\textit{single image super resolution}, ở đó, hệ thống cần tạo ra một bức ảnh có độ
phân giải cao dựa
trên một ảnh có độ phân giải thấp hơn. Khi đó, việc dự đoán giá trị của các
pixel trong ảnh đầu ra là một bài toán regression với nhiều đầu ra. 

% \subsection{Transcription}
% Trong loại bài toán này, hệ thống machine learning được yêu cầu quan sát một
% loại dữ liệu và 



\subsection{Machine translation}

Trong bài toán này, đầu vào là một câu, đoạn, hay bài văn trong một ngôn ngữ,
và
chương trình máy tính được yêu cầu chuyển đổi nó sang một ngôn ngữ khác. Lời
giải cho bài toán này gần đây đã có nhiều bước phát triển vượt bậc dựa trên các
thuật toán deep learning.


\subsection{Clustering}
\textit{Clustering} là bài toán \textit{phân nhóm} toàn bộ dữ liệu $\mathcal{X}$
thành các nhóm nhỏ dựa trên
sự liên quan giữa các dữ liệu trong mỗi nhóm. 

\textbf{Ví dụ:} phân nhóm khách hàng dựa
trên hành vi mua hàng. Điều này cũng giống như việc ta đưa cho một đứa trẻ rất
nhiều mảnh ghép với các hình thù và màu sắc khác nhau, ví dụ tam giác, vuông,
tròn với màu xanh và đỏ, sau đó yêu cầu trẻ phân chúng thành từng nhóm. Mặc dù
không cho trẻ biết mảnh nào tương ứng với hình nào hoặc màu nào, nhiều khả năng
chúng vẫn có thể phân loại các mảnh ghép theo màu hoặc hình dạng.

\subsection{Completion}
\textit{Completion} là bài toán \textit{điền} những giá trị còn thiếu của một
điểm dữ liệu. Trong nhiều bài toán thực tế, việc thu thập toàn bộ thông tin của
một điểm dữ liệu, ví dụ khách hàng, là không khả thi. Nhiệm vụ của bài toán này
là dựa trên mối tương quan giữa các điểm dữ liệu để dự đoán những giá trị còn
thiếu. \textit{Các hệ thống khuyến nghị} (\textit{recommendation system}) là
một ví
dụ điển hình của loại này. 

Bạn đọc có thể đọc thêm về các bài toán \textit{xếp hạng} (\textit{ranking}),
\textit{thu thập thông tin} (\textit{information retrieval}), \textit{giảm
nhiễu} (\textit{denoising}), v.v..


% \subsection{Machine translation}

% \textbf{Ví dụ 2:} Thuật toán dò các khuôn mặt trong một bức ảnh đã được phát
% triển từ rất lâu. Thời gian đầu, facebook sử dụng thuật toán này để chỉ ra các
% khuôn mặt trong một bức ảnh và yêu cầu người dùng \textit{tag friends} - tức gán
% nhãn cho mỗi khuôn mặt. Số lượng cặp dữ liệu (\textit{khuôn mặt, tên người})
% càng lớn, độ chính xác ở những lần tự động \textit{tag} tiếp theo sẽ càng lớn.

% \textbf{Ví dụ 3:} Bản thân thuật toán dò tìm các khuôn mặt trong 1 bức ảnh cũng
% là một thuật toán Supervised learning với training data (dữ liệu học) là hàng
% ngàn cặp (\textit{ảnh, mặt người}) và (\textit{ảnh, không phải mặt người}) được
% đưa vào. Chú ý là dữ liệu này chỉ phân biệt \textit{mặt người} và \textit{không
% phải mặt ngưòi} mà không phân biệt khuôn mặt của những người khác nhau.



% \end{itemize}


\section{Phép đánh giá, $P$}

\index{training set--tập huấn luyện}
\index{test set--tập kiểm thử}

Để kiểm tra năng lực của một thuật toán machine learning, chúng ta cần phải
thiết kế các phép đánh giá có thể đo đạc được kết quả. 

Thông thường, khi thực hiện một thuật toán machine learning, dữ liệu sẽ được
chia thành hai phần riêng biệt: \textit{tập huấn luyện} (\textit{training set})
và \textit{tập kiểm thử} (\textit{test set}). Tập huấn luyện sẽ được dùng để tìm
các tham số mô hình. Tập kiểm thử được dùng để đánh giá năng lực của mô hình tìm
được. Có một điểm cần lưu ý rằng khi tìm các tham số mô hình, ta chỉ được dùng
các thông tin trong tập huấn luyện. Việc đánh giá có thể được áp dụng lên cả
hai tập hợp. Muốn mô hình thực hiện tốt trên tập kiểm thử thì nó trước hết phải
hoạt động tốt trên tập huấn luyện. 

\index{online learning}
\index{offline learning}
\textbf{Lưu ý:} Ranh giới giữa tập huấn luyện và tập kiểm thử đôi khi không rõ
ràng. Các thuật toán thực tế liên tục được cập nhật dựa trên dữ liệu mới thêm
vào, các thuật toán này được gọi là \textit{online learning} hoặc \textit{online
training}. Phần dữ liệu mới này ban đầu không được hệ thống sử dụng để xây
dựng mô hình, nhưng về sau có thể được mô hình sử dụng để cải tiến. Ngược với
\textit{online learning} là \textit{offline learning}, ở đó hệ thống xây dựng mô
hình \textit{một lần} dựa trên một tập chính là tập huấn luyện. Các điểm dữ
liệu không được dùng trong quá trình xây dựng hệ thống được coi là tập kiểm thử.
Trong cuốn sách này, khi không đề cập gì thêm, các thuật toán được ngầm hiểu là
\textit{offline
learning}, trong đó \textit{training set} là tập hợp được dùng để xây dựng mô
hình ban
đầu, \textit{test set} là tập hợp được dùng để đánh giá hiệu quả của mô hình
được xây dựng đó.

\section{Kinh nghiệm, $E$}



Việc huấn luyện các mô hình machine learning có thể coi là việc cho chúng
\textit{trải nghiệm} trên các \textit{tập dữ liệu} (\textit{dataset}) -- chính là
\textit{training set}. Các tập dữ liệu khác nhau sẽ cho các mô hình các trải
nghiệm khác nhau. Chất lượng của
các tập dữ liệu này cũng ảnh hưởng tới hiệu năng của mô hình. 

Dựa trên tính chất của các tập dữ liệu, các thuật toán machine learning có thể
phân loại thành hai nhóm chính là
\textit{học không giám sát} (\textit{unsupervised learning}) và \textit{học có
giám sát} (\textit{supervised learning}).

\index{supervised learning -- học có giám sát}
\textbf{Supervised learning} là thuật toán dự đoán đầu ra của một hoặc nhiều dữ
liệu mới dựa trên các cặp (\textit{đầu vào, đầu ra}) đã biết từ trước. 
Supervised learning là nhóm phổ biến nhất trong các thuật toán machine learning.

Một cách toán học, supervised learning là khi có một tập hợp biến đầu
vào $ \mathcal{X} = \{\mathbf{x}_1, \mathbf{x}_2, \dots, \mathbf{x}_N\} $ và một
tập hợp đầu ra tương ứng $ \mathcal{Y} = \{\mathbf{y}_1, \mathbf{y}_2, \dots,
\mathbf{y}_N\} $, trong đó $ \mathbf{x}_i, \mathbf{y}_i $ là các vector. Các cặp
dữ liệu biết trước $ (\mathbf{x}_i, \mathbf{y}_i) \in \mathcal{X} \times
\mathcal{Y} $ tạo nên tập huấn luyện. Từ
tập huấn luyện này, chúng ta cần tạo ra một hàm số ánh xạ mỗi phần tử từ tập
$\mathcal{X}$ sang một phần tử (xấp xỉ) tương ứng của tập $\mathcal{Y}$: $$
\mathbf{y}_i \approx f(\mathbf{x}_i), ~~ \forall i = 1, 2, \dots, N$$ Mục đích
là xấp xỉ hàm số $f$ thật tốt để khi có một dữ liệu $\mathbf{x}$ mới, chúng ta
có thể tính được nhãn tương ứng của nó $ \mathbf{y} = f(\mathbf{x})$. 


\index{unsupervised learning -- học không giám sát} 
Ngược lại, trong \textbf{unsupervised learning}, chúng ta không biết được kết
quả đầu ra mà chỉ biết các vector đặc trưng của dữ liệu đầu vào. Các thuật toán
unsupervised learning sẽ dựa vào cấu trúc của dữ liệu để thực hiện một công việc
nào đó, ví dụ như phân nhóm hoặc \textit{giảm số chiều của dữ liệu}
(\textit{dimentionality reduction}). Một cách toán học, unsupervised learning là
khi chúng ta chỉ có dữ liệu đầu vào $\mathcal{X}$ mà không biết đầu ra 
$\mathcal{Y}$ tương ứng.

Không giống như trong supervised learning, chúng ta không biết câu trả lời chính
xác cho mỗi dữ liệu đầu vào trong unsupervised learning. Giống như khi ta học,
ta chỉ được đưa cho một chữ cái mà không nói đó là chữ A hay chữ B. Cụm từ
\textit{không giám sát}, hay \textit{không ai chỉ bảo} (\textit{unsupervised})
được đặt tên theo nghĩa này.

Từ góc độ xác suất thống kê, unsupervised learning trải nghiệm qua rất nhiều ví
dụ (các điểm dữ liệu) $\bx$ và cố gắng học phân phối xác suất $p(\bx)$ hoặc các
tính chất của phân phối đó một cách trực tiếp hoặc gián tiếp. Trong khi đó,
supervised learning quan sát các ví dụ $\bx$ và các kết quả tương ứng $\by$, sau
đó cố gắng học cách dự đoán $\by$ từ $\bx$ thông qua việc đánh giá xác suất có
điều kiện $p(\by |\bx)$. Xác suất này có thể diễn đạt bằng lời là biết rằng một
điểm dữ liệu có vector đặc trưng là $\bx$, xác suất để đầu ra của nó bằng $\by$
là bao nhiêu.

Ranh giới giữa unsupervised learning và supervised learning đôi khi là không rõ
ràng. Thông thường, người ta thường coi các bài classification, regression là
supervised learning, các bài clustering hay \textit{density estimation}
(\textit{ước lượng một phân phối}) là unsupervised learning. 

\index{semi-supervised learning -- học bán giám sát}
Có những bài toán mà dữ liệu được dùng để huấn luyện bao gồm cả những dữ liệu có
nhãn và chưa được gán nhãn. Các bài toán khi chúng ta có
một lượng lớn dữ liệu $\mathcal{X}$ nhưng chỉ một phần trong chúng được gán nhãn
được gọi là \textit{học bán giám sát}, hay \textbf{semi-supervised learning}.
Những bài toán thuộc nhóm này nằm giữa hai nhóm được nêu bên trên.

Một ví dụ điển hình của nhóm này là chỉ có một phần ảnh hoặc văn bản được gán
nhãn (ví dụ bức ảnh về người, động vật hoặc các văn bản khoa học, chính trị) và
phần lớn các bức ảnh/văn bản khác chưa được gán nhãn được thu thập từ internet.
Thực tế cho thấy rất nhiều các bài toán machine learning thuộc vào nhóm này vì
việc thu thập dữ liệu có nhãn tốn rất nhiều thời gian và có chi phí cao. Rất
nhiều loại dữ liệu, ví dụ như ảnh y học, thậm chí cần phải có chuyên gia mới gán
nhãn được. Ngược lại, dữ liệu chưa có nhãn có thể được thu thập với chi phí thấp
từ internet.

\index{reinforcement learning - học củng cố}
Có những thuật toán machine learning không trải nghiệm trên một tập dữ liệu
cố định. Ví dụ, \textit{học củng cố} (\textbf{reinforcement learning}) trải
nghiệm trực tiếp với môi trường xung quanh, liên tục nhận phản hồi từ môi trường
để tự cải thiện hành vi của hệ thống trong các môi trường mới. Các ví dụ điển
hình của reinforcement learning là việc huấn luyện cho xe tự lái dựa vào ảnh
nhận từ camera và điều khiển tay lái cũng như tốc độ của xe. Reinforcement
learning hiện nay chủ yếu được áp dụng vào các trò chơi, khi mà máy tính có thể
mô phỏng được các trạng thái của môi trường và huấn luyện thuật toán thông qua
rất nhiều vòng lặp.

\textbf{Ví dụ 1:} AlphaGo gần đây nổi tiếng với việc chơi cờ vây thắng cả con
người (\url{https://goo.gl/PzKcvP}). {Cờ vây được xem là có độ phức tạp cực kỳ
cao}\footnote{\textit{Google DeepMind's AlphaGo: How it
works} (\url{https://goo.gl/nDNcCy}).} với tổng số nước đi là xấp xỉ $10^{761}
$,
so với cờ vua là $10^{120} $ và tổng số nguyên tử trong toàn vũ trụ là khoảng
$10^{80}$!! Hệ thống phải chọn ra một \textit{đường đi nước bước} tối ưu trong
số hàng nhiều tỉ tỉ lựa chọn, và tất nhiên, việc thử tất cả các lựa chọn là
không khả thi. Về cơ bản, AlphaGo bao gồm các thuật toán thuộc cả supervised
learning và reinforcement learning. Trong phần supervised learning, dữ liệu từ
các ván cờ do con người chơi với nhau được đưa vào để huấn luyện. Tuy nhiên, mục
đích cuối cùng của AlphaGo không phải là chơi như con người mà phải thậm chí
thắng cả con người. Vì vậy, sau khi \textit{học} xong các ván cờ của con người,
AlphaGo tự chơi với chính nó với hàng triệu ván chơi để tìm ra các nước đi mới
tối ưu hơn. Thuật toán trong phần tự chơi này được xếp vào loại reinforcement
learning.

Gần đây, Google DeepMind đã tiến thêm một bước đáng kể với AlphaGo Zero. Hệ
thống này thậm chí không cần học từ các ván cờ của con người. Nó có thể tự chơi
với chính mình để tìm ra các chiến thuật tối ưu. Sau 40 ngày được huấn
luyện, nó đã thắng tất cả các con người và hệ thống khác, bao gồm
AlphaGo\footnote{\textit{AlphaGo Zero:
Learning from scratch} (\url{https://goo.gl/xtDjoF}).}.

\textbf{Ví dụ 2:} {Huấn luyện cho máy tính chơi game
Mario}\footnote{\textit{MarI/O - Machine Learning for Video Games}
(\url{https://goo.gl/QekkRz})}. Đây là một chương trình thú vị dạy máy tính chơi
game Mario. Game này đơn giản hơn cờ vây vì tại một thời điểm, người chơi chỉ
phải bấm một số lượng nhỏ các nút (di chuyển, nhảy, bắn đạn) hoặc không cần bấm
nút nào. Đồng thời, phản ứng của máy cũng đơn giản hơn và lặp lại ở mỗi lần chơi
(tại thời điểm cụ thể sẽ xuất hiện một chướng ngại vật cố định ở một vị trí cố
định). Đầu vào của thuật toán là sơ đồ của màn hình tại thời điểm hiện tại,
nhiệm vụ của thuật toán là với đầu vào đó, tổ hợp phím nào nên được bấm. Việc
huấn luyện này được dựa trên điểm số cho việc di chuyển được bao xa trong thời
gian bao lâu trong game, càng xa và càng nhanh thì được điểm thưởng càng cao
(điểm thưởng này không phải là điểm của trò chơi mà là điểm do chính người lập
trình tạo ra). Thông qua huấn luyện, thuật toán sẽ tìm ra một cách tối đa số
điểm trên, qua đó đạt được mục đích cuối cùng là cứu công chúa.

Reinforcement learning là một lĩnh vực thú vị trong machine learning. Rất tiếc,
reinforcement learning nằm ngoài phạm vi của cuốn sách này.

\section{Hàm mất mát và tham số mô hình}
\index{model parameters}
\index{loss function -- hàm mất mát}
\index{model parameter -- tham số mô hình}
Mỗi mô hình machine learning được mô tả bởi \textit{các tham số mô hình}
(\textit{model parameters}). Công việc của một thuật toán machine learning là đi
tìm các tham số mô hình phù hợp với mỗi bài toán. Việc đi tìm các tham số mô
hình có liên quan mật thiết đến các phép đánh giá. Mục đích của chúng ta là đi
tìm các tham số mô hình sao cho các phép đánh giá cho kết quả tốt nhất. Trong
bài toán classification, kết quả tốt có thể được hiểu là ít điểm dữ liệu bị phân
lớp sai nhất. Trong bài toán regression, kết quả tốt là khi sự sai lệch giữa đầu
ra dự đoán và đầu ra thực sự là ít nhất.

Quan hệ giữa một phép đánh giá và các tham số mô hình thường được mô tả thông
qua một hàm số được gọi là \textit{hàm mất mát} (\textit{loss function},
hay \textit{cost function}). Hàm mất mát này thường có giá trị nhỏ khi phép
đánh giá cho kết quả tốt và ngược lại. Việc đi tìm các tham số mô hình sao cho
phép đánh giá trả về kết quả tốt tương đương với việc tối thiểu hàm mất mát.
Như vậy, việc xây dựng một mô hình machine learning chính là việc đi giải một
bài toán tối ưu. Quá trình đó có thể được coi là quá trình
\textit{learning} của \textit{machine}. 

Tập hợp các tham số mô hình thường được ký hiệu bằng $\theta$, hàm mất mát của
mô hình thường được ký hiệu là $\mathcal{L}(\theta)$ hoặc $J(\theta)$. Bài
toán tối thiểu hàm mất mát để tìm tham số mô hình thường được viết dưới dạng:
\begin{equation}
    \theta^* = \argmin_{\theta}\mathcal{L}(\theta)
\end{equation}
ký hiệu $\displaystyle \argmin_{\theta}\L(\theta)$ được hiểu là giá trị của
$\theta$ để hàm số
$\L(\theta)$ đạt giá trị nhỏ nhất. Khi sử dụng $\argmin$, chúng ta phải
chỉ rõ nó được thực hiện theo các biến số nào bằng cách ghi các biến số ở dưới
$\min$ (ở đây là $\theta$). Nếu hàm số chỉ có một biến số, ta có thể bỏ qua
biến số đó dưới $\min$. Tuy nhiên, biến số nên được ghi rõ ràng để giảm
thiểu sự nhầm lẫn. $\argmax$ cũng được sử dụng một cách tương tự khi ta cần
tìm giá trị của các biến số để một hàm số đạt giá trị lớn nhất. 

Một hàm số $\L(\theta)$ bất kỳ có thể có rất nhiều giá trị của $\theta$ để nó
đạt giá trị nhỏ nhất, hoặc cũng có thể nó không chặn dưới. Thậm chí, việc tìm
giá trị nhỏ nhất của một hàm số đôi khi là không khả thi. Trong machine learning
cũng như nhiều bài toán tối ưu thực tế, việc chỉ cần tìm ra một bộ tham số
$\theta$ làm cho hàm mất mát đạt giá trị nhỏ nhất, hoặc thậm chí đạt một giá
trị cực tiểu\footnote{Lưu ý rằng cực tiểu trong toán học không có nghĩa là nhỏ
nhất.}, thường mang lại các kết quả khả quan. 

Để hiểu rõ bản chất của các thuật toán machine learning, việc nắm vững các kỹ
thuật tối ưu cơ bản là rất quan trọng. Cuốn sách này có nhiều chương cung cấp
các kiến thức cần thiết cho tối ưu, bao gồm tối ưu không ràng buộc
(Chương~\ref{cha:gradient_descent}) và tối ưu có ràng buộc (xem
Phần~\ref{part:cvxopt}). 

Trong các chương tiếp theo của cuốn sách này, chúng ta sẽ dần làm quen với các
thành phần cơ bản của một hệ thống machine learning. 

\section{Tài liệu tham khảo}
[1] Mục 5.1, \href{http://www.deeplearningbook.org/}{\textit{Deep learning}} (Goodfellow, 2016) \cite{Goodfellow-et-al-2016}.

% \begin{mydeff}
%     kasdj 
% \end{mydeff}

% Có hai cách phổ biến phân nhóm các thuật toán Machine learning. Một là dựa trên
% phương thức học (learning style), hai là dựa trên chức năng (function) (của mỗi
% thuật toán).


% \section{Phân nhóm dựa trên phương thức học}

% Theo phương thức học, các thuật toán Machine Learning thường được chia làm 4
% nhóm: Supervise learning, Unsupervised learning, Semi-supervised lerning và
% Reinforcement learning. \textit{Có một số cách phân nhóm không có
% Semi-supervised learning hoặc Reinforcement learning.}

% \index{Supervised Learning}
% \subsection{Supervised Learning (Học có giám sát) }
% Supervised learning là thuật toán dự đoán đầu ra (outcome) của một dữ liệu mới
% (new input) dựa trên các cặp (\textit{input, outcome}) đã biết từ trước. Cặp dữ
% liệu này còn được gọi là (\textit{data. label}), tức (\textit{dữ liệu, nhãn}).
% Supervised learning là nhóm phổ biến nhất trong các thuật toán Machine Learning.

% Một cách toán học, Supervised learning là khi chúng ra có một tập hợp biến đầu
% vào $ \mathcal{X} = \{\mathbf{x}_1, \mathbf{x}_2, \dots, \mathbf{x}_N\} $ và một
% tập hợp nhãn tương ứng $ \mathcal{Y} = \{\mathbf{y}_1, \mathbf{y}_2, \dots,
% \mathbf{y}_N\} $, trong đó $ \mathbf{x}_i, \mathbf{y}_i $ là các vector. Các cặp
% dữ liệu biết trước $ (\mathbf{x}_i, \mathbf{y}_i) \in \mathcal{X} \times
% \mathcal{Y} $ được gọi là tập \textit{training data} (dữ liệu huấn luyện). Từ
% tập traing data này, chúng ta cần tạo ra một hàm số ánh xạ mỗi phần tử từ tập
% $\mathcal{X}$ sang một phần tử (xấp xỉ) tương ứng của tập $\mathcal{Y}$: $$
% \mathbf{y}_i \approx f(\mathbf{x}_i), ~~ \forall i = 1, 2, \dots, N$$ Mục đích
% là xấp xỉ hàm số $f$ thật tốt để khi có một dữ liệu $\mathbf{x}$ mới, chúng ta
% có thể tính được nhãn tương ứng của nó $ \mathbf{y} = f(\mathbf{x}) $.

% \textbf{Ví dụ 1:} trong nhận dạng chữ viết tay (Hình
% \ref{fig:categories_mnist}), ta có ảnh của hàng nghìn ví dụ của mỗi chữ số được
% viết bởi nhiều người khác nhau. Chúng ta đưa các bức ảnh này vào trong một thuật
% toán và chỉ cho nó biết mỗi bức ảnh tương ứng với chữ số nào. Sau khi thuật toán
% tạo ra (sau khi \textit{học}) một mô hình, tức một hàm số mà đầu vào là một bức
% ảnh và đầu ra là một chữ số, khi nhận được một bức ảnh mới mà mô hình
% \textbf{chưa nhìn thấy bao giờ}, nó sẽ dự đoán bức ảnh đó chứa chữ số nào.


% Ví dụ này khá giống với cách học của con người khi còn nhỏ. Ta đưa bảng chữ cái
% cho một đứa trẻ và chỉ cho chúng đây là chữ A, đây là chữ B. Sau một vài lần
% được dạy thì trẻ có thể nhận biết được đâu là chữ A, đâu là chữ B trong một cuốn
% sách mà chúng chưa nhìn thấy bao giờ.

% \textbf{Ví dụ 2:} Thuật toán dò các khuôn mặt trong một bức ảnh đã được phát
% triển từ rất lâu. Thời gian đầu, facebook sử dụng thuật toán này để chỉ ra các
% khuôn mặt trong một bức ảnh và yêu cầu người dùng \textit{tag friends} - tức gán
% nhãn cho mỗi khuôn mặt. Số lượng cặp dữ liệu (\textit{khuôn mặt, tên người})
% càng lớn, độ chính xác ở những lần tự động \textit{tag} tiếp theo sẽ càng lớn.

% \textbf{Ví dụ 3:} Bản thân thuật toán dò tìm các khuôn mặt trong 1 bức ảnh cũng
% là một thuật toán Supervised learning với training data (dữ liệu học) là hàng
% ngàn cặp (\textit{ảnh, mặt người}) và (\textit{ảnh, không phải mặt người}) được
% đưa vào. Chú ý là dữ liệu này chỉ phân biệt \textit{mặt người} và \textit{không
% phải mặt ngưòi} mà không phân biệt khuôn mặt của những người khác nhau.

% Thuật toán supervised learning còn được tiếp tục chia nhỏ ra thành hai loại
% chính:

% \index{Classification problems}
% \subsubsection{Classification (Phân loại)}
%  Một bài toán được gọi là \textit{classification} nếu các \textit{label} của
%  \textit{input data} được chia thành một số hữu hạn nhóm. Ví dụ: Gmail xác định
%  xem một email có phải là spam hay không; các hãng tín dụng xác định xem một
%  khách hàng có khả năng thanh toán nợ hay không. Ba ví dụ phía trên được chia
%  vào loại này.

% \index{Regression problems}
% \subsubsection{Regression (Hồi quy)}
% (tiếng Việt dịch là \textit{Hồi quy}, tôi không thích cách dịch này vì bản thân
% không hiểu nó nghĩa là gì)

% Nếu \textit{label} không được chia thành các nhóm mà là một giá trị thực cụ thể.
% Ví dụ: một căn nhà rộng $x ~ \text{m}^2$, có $y$ phòng ngủ và cách trung tâm
% thành phố $z~ \text{km}$ sẽ có giá là bao nhiêu?

% Gần đây \href{http://how-old.net/}{Microsoft có một ứng dụng dự đoán giới tính
% và tuổi dựa trên khuôn mặt}. Phần dự đoán giới tính có thể coi là thuật toán
% \textbf{Classification}, phần dự đoán tuổi có thể coi là thuật toán
% \textbf{Regression}. \textit{Chú ý rằng phần dự đoán tuổi cũng có thể coi là
% \textbf{Classification} nếu ta coi tuổi là một số nguyên dương không lớn hơn
% 150, chúng ta sẽ có 150 class (lớp) khác nhau.}

% \index{Unsupervised Learning}
% \subsection{Unsupervised Learning (Học không giám sát)}
% Trong thuật toán này, chúng ta không biết được \textit{outcome} hay
% \textit{nhãn} mà chỉ có dữ liệu đầu vào. Thuật toán unsupervised learning sẽ dựa
% vào cấu trúc của dữ liệu để thực hiện một công việc nào đó, ví dụ như phân nhóm
% (clustering) hoặc giảm số chiều của dữ liệu (dimention reduction) để thuận tiện
% trong việc lưu trữ và tính toán. Một cách toán học, Unsupervised learning là khi
% chúng ta chỉ có dữ liệu vào
% $\mathcal{X} $ mà không biết \textit{nhãn} $\mathcal{Y}$ tương ứng.

% Những thuật toán loại này được gọi là Unsupervised learning vì không giống như
% Supervised learning, chúng ta không biết câu trả lời chính xác cho mỗi dữ liệu
% đầu vào. Giống như khi ta học, không có thầy cô giáo nào chỉ cho ta biết đó là
% chữ A hay chữ B. Cụm \textit{không giám sát} (hoặc \textit{không ai chỉ bảo})
% được đặt tên theo nghĩa này.

% Các bài toán Unsupervised learning được tiếp tục chia nhỏ thành hai loại:

% \index{Clustering problems}
% \subsubsection{Clustering (phân nhóm)}
% Một bài toán phân nhóm toàn bộ dữ liệu $\mathcal{X}$ thành các nhóm nhỏ dựa trên
% sự liên quan giữa các dữ liệu trong mỗi nhóm. Ví dụ: phân nhóm khách hàng dựa
% trên hành vi mua hàng. Điều này cũng giống như việc ta đưa cho một đứa trẻ rất
% nhiều mảnh ghép với các hình thù và màu sắc khác nhau, ví dụ tam giác, vuông,
% tròn với màu xanh và đỏ, sau đó yêu cẩu trẻ phân chúng thành từng nhóm. Mặc dù
% không cho trẻ biết mảnh nào tương ứng với hình nào hoặc màu nào, nhiều khả năng
% chúng vẫn có thể phân loại các mảnh ghép theo màu hoặc hình dạng.

% \index{Association problems}
% \subsubsection{Association}
% Là bài toán khi chúng ta muốn khám phá ra một quy luật dựa trên nhiều dữ liệu
% cho trước. Ví dụ: những khách hàng nam mua quần áo thường có xu hướng mua thêm
% đồng hồ hoặc thắt lưng; những khán giả xem phim Spider Man thường có xu hướng
% xem thêm phim Bat Man, dựa vào đó tạo ra một hệ thống gợi ý khách hàng
% (Recommendation System), thúc đẩy nhu cầu mua sắm.


% \index{Semi-Supervised Learning}
% \subsection{Semi-Supervised Learning (Học bán giám sát)}
% Các bài toán khi chúng ta có một lượng lớn dữ liệu $\mathcal{X}$ nhưng chỉ một
% phần trong chúng được gán nhãn được gọi là Semi-Supervised Learning. Những bài
% toán thuộc nhóm này nằm giữa hai nhóm được nêu bên trên.

% Một ví dụ điển hình của nhóm này là chỉ có một phần ảnh hoặc văn bản được gán
% nhãn (ví dụ bức ảnh về người, động vật hoặc các văn bản khoa học, chính trị) và
% phần lớn các bức ảnh/văn bản khác chưa được gán nhãn được thu thập từ internet.
% Thực tế cho thấy rất nhiều các bài toàn Machine Learning thuộc vào nhóm này vì
% việc thu thập dữ liệu có nhãn tốn rất nhiều thời gian và có chi phí cao. Rất
% nhiều loại dữ liệu thậm chí cần phải có chuyên gia mới gán nhãn được (ảnh y học
% chẳng hạn). Ngược lại, dữ liệu chưa có nhãn có thể được thu thập với chi phí
% thấp từ internet.


% \index{Reinforcement Learning}
% \subsection{Reinforcement Learning (Học Củng Cố)}
% Reinforcement learning là các bài toán giúp cho một hệ thống tự động xác định
% hành vi dựa trên hoàn cảnh để đạt được lợi ích cao nhất (maximizing the
% performance). Hiện tại, Reinforcement learning chủ yếu được áp dụng vào Lý
% Thuyết Trò Chơi (Game Theory), các thuật toán cần xác định nưóc đi tiếp theo để
% đạt được điểm số cao nhất.

% % <div class="imgcap"> <div > <a href = "/2016/12/27/categories/"> <img
% % src="/assets/categories/alphago.jpeg" width = "800"></a> </div> <div
% % class="thecap">AlphaGo chơi cờ vây với Lee Sedol. AlphaGo là một ví dụ của
% % Reinforcement learning. <br> (Nguồn: <a href ="http://www.tomshardware.com/new
% % s/alphago-defeats-sedol-second-time,31377.html">AlphaGo AI Defeats Sedol
% % Again, With 'Near Perfect Game')</a></div> </div>

%  % \begin{figure}
%  %   \centering
%  %   \includegraphics[width = .7\textwidth]{../categories/alphago.jpeg}
%  %   \caption{AlphaGo chơi cờ vây với Lee Sedol. AlphaGo là một ví dụ của
%  %   Reinforcement learning (Nguồn \href{http://www.tomshardware.com/news/alphag
%  %   o-defeats-sedol-second-time,31377.html}{AlphaGo AI Defeats Sedol Again,
%  %   With 'Near Perfect Game'})}
%  %   \label{fig:categories_alphago}
%  % \end{figure}

% \textbf{Ví dụ 1:} AlphaGo
% \href{https://gogameguru.com/tag/deepmind-alphago-lee-sedol/}{gần đây nổi tiếng
% với việc chơi cờ vây thắng cả con người}.
% \href{https://www.tastehit.com/blog/google-deepmind-alphago-how-it-works/}{Cờ
% vây được xem là có độ phức tạp cực kỳ cao} với tổng số nước đi là xấp xỉ
% $10^{761} $, so với cờ vua là $10^{120} $ và tổng số nguyên tử trong toàn vũ trụ
% là khoảng $10^{80}$!! Vì vậy, thuật toán phải chọn ra 1 nước đi tối ưu trong số
% hàng nhiều tỉ tỉ lựa chọn, và tất nhiên, không thể áp dụng thuật toán tương tự
% như \href{https://en.wikipedia.org/wiki/Deep_Blue_(chess_computer}{IBM Deep
% Blue}) (IBM Deep Blue đã thắng con người trong môn cờ vua 20 năm trước). Về cơ
% bản, AlphaGo bao gồm các thuật toán thuộc cả Supervised learning và
% Reinforcement learning. Trong phần Supervised learning, dữ liệu từ các ván cờ do
% con người chơi với nhau được đưa vào để huấn luyện. Tuy nhiên, mục đích cuối
% cùng của AlphaGo không phải là chơi như con người mà phải thậm chí thắng cả con
% người. Vì vậy, sau khi \textit{học} xong các ván cờ của con người, AlphaGo tự
% chơi với chính nó với hàng triệu ván chơi để tìm ra các nước đi mới tối ưu hơn.
% Thuật toán trong phần tự chơi này được xếp vào loại Reinforcement learning. (Xem
% thêm tại \href{https://www.tastehit.com/blog/google-deepmind-alphago-how-it-work
% s/}{Google DeepMind's AlphaGo: How it works}).


% \textbf{Ví dụ 2:} \href{https://www.youtube.com/watch?v=qv6UVOQ0F44}{Huấn luyện
% cho máy tính chơi game Mario}. Đây là một chương trình thú vị dạy máy tính chơi
% game Mario. Game này đơn giản hơn cờ vây vì tại một thời điểm, người chơi chỉ
% phải bấm một số lượng nhỏ các nút (di chuyển, nhảy, bắn đạn) hoặc không cần bấm
% nút nào. Đồng thời, phản ứng của máy cũng đơn giản hơn và lặp lại ở mỗi lần chơi
% (tại thời điểm cụ thể sẽ xuất hiện một chướng ngại vật cố định ở một vị trí cố
% định). Đầu vào của thuật toán là sơ đồ của màn hình tại thời điểm hiện tại,
% nhiệm vụ của thuật toán là với đầu vào đó, tổ hợp phím nào nên được bấm. Việc
% huấn luyện này được dựa trên điểm số cho việc di chuyển được bao xa trong thời
% gian bao lâu trong game, càng xa và càng nhanh thì được điểm thưởng càng cao
% (điểm thưởng này không phải là điểm của trò chơi mà là điểm do chính người lập
% trình tạo ra). Thông qua huấn luyện, thuật toán sẽ tìm ra một cách tối ưu để tối
% đa số điểm trên, qua đó đạt được mục đích cuối cùng là cứu công chúa.



% % \section{Phân nhóm dựa trên chức năng }

% % Có một cách phân nhóm thứ hai dựa trên chức năng của các thuật toán. Trong phần này, tôi xin chỉ liệt kê các thuật toán. Thông tin cụ thể sẽ được trình bày trong các bài viết khác tại blog này. Trong quá trình viết, tôi có thể sẽ thêm bớt một số thuật toán.


% % \subsection{Regression Algorithms}
% % \begin{enumerate}
% %   \item \href{http://machinelearningcoban.com/2016/12/28/linearregression/}{Linear Regression}

% %   \item \href{http://machinelearningcoban.com/2017/01/27/logisticregression/#sigmoid-function}{Logistic Regression}


% %   \item Stepwise Regression
% % \end{enumerate}


% % \subsection{Classification Algorithms }

% % \begin{enumerate}
% %   \item Linear Classifier

% %   \item  Support Vector Machine (SVM)

% %   \item  Kernel SVM

% %   \item Sparse Represntation-based classification (SRC)
% % \end{enumerate}


% % \subsection{Instance-based Algorithms }

% %  \begin{enumerate}
% %    \item \href{http://machinelearningcoban.com/2017/01/08/knn/}{k-Nearest Neighbor (kNN)}

% %    \item Learnin Vector Quantization (LVQ)
% %  \end{enumerate}


% % \subsection{Regularization Algorithms }

% %  \begin{enumerate}
% %    \item  Ridge Regression

% %    \item  Least Absolute Shrinkage and Selection Operator (LASSO)

% %    \item  Least-Angle Regression (LARS)
% %  \end{enumerate}


% % \subsection{Bayesian Algorithms}
% %  \begin{enumerate}
% %    \item Naive Bayes

% %    \item Gaussian Naive Bayes
% %  \end{enumerate}


% % \subsection{Clustering Algorithms}
% % \begin{enumerate}
% %   \item \href{http://machinelearningcoban.com/2017/01/01/kmeans/}{k-Means clustering}

% %   \item k-Medians

% %   \item Expectation Maximization (EM)
% % \end{enumerate}


% % \subsection{Artificial Neural Network Algorithms }
% % \begin{enumerate}
% %   \item \href{http://machinelearningcoban.com/2017/01/21/perceptron/}{Perceptron}

% %   \item \href{http://machinelearningcoban.com/2017/02/17/softmax/}{Softmax Regression}

% %   \item  \href{http://machinelearningcoban.com/2017/02/24/mlp/}{Multi-layer Perceptron}

% %   \item \href{http://machinelearningcoban.com/2017/02/24/mlp/#-backpropagation}{Back-Propagation }

% % \end{enumerate}

% % \subsection{Dimensionality Reduction Algorithms }
% % \begin{enumerate}
% %   \item Principal Component Analysis (PCA)

% %   \item Linear Discriminant Analysis (LDA)

% % \end{enumerate}

% % \subsection{Ensemble Algorithms }
% %  \begin{enumerate}
% %    \item Boosting

% %    \item AdaBoost

% %   \item Random Forest
% %  \end{enumerate}

% % Và còn rất nhiều các thuật toán khác.


% \section{Tài liệu tham khảo }
% \begin{enumerate}
%   \item \href{http://machinelearningmastery.com/a-tour-of-machine-learning-algorithms/}{A Tour of Machine Learning Algorithms}

%   \item \href{https://ongxuanhong.wordpress.com/2015/10/22/diem-qua-cac-thuat-toan-machine-learning-hien-dai/}{Điểm qua các thuật toán Machine Learning hiện đại}


% \end{enumerate}



% % \begin{mynote}%{Tiêu đề ở đây}
% %    \begin{itemize}
% %      \item asdf
% %      \item ;alsdf
% %    \end{itemize}
% % \end{mynote}




% % \begin{myfr}
% %           This is the text of the theorem. The counter is automatically assigned and,
% %           in this example, prefixed with the section number. This theorem is numbered with
% %           , it is given on page,
% %           and it is titled.
% %           \begin{itemize}
% %             \item lajsdf

% %             \item ljasljdf
% %           \end{itemize}
% % \end{myfr}

% % \begin{mytheo}{aa}{jasdf}
% %     lkasjlf
% % \end{mytheo}

% % \begin{mydef}{halfspace - nửa không gian}{theoexample}
% %           This is the text of the theorem. The counter is automatically assigned and,
% %           in this example, prefixed with the section number. This theorem is numbered with
% %           , it is given on page,
% %           and it is titled.
% % \end{mydef}
% % \ref{def:theoexample}
% % % \ref{}
% % cyan
% % \begin{myalg}{K-means Clustering}{theoexample}
% %           This is the text of the theorem. The counter is automatically assigned and,
% %           in this example, prefixed with the section number. This theorem is numbered with
% %           , it is given on page,
% %           and it is titled.
% % \end{myalg}
